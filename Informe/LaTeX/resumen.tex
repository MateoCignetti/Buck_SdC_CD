\section*{\large{RESUMEN}}
\vspace{-0.35cm}
\justifying

En este trabajo práctico se desarrolla un sistema de control sobre un convertidor buck en carácter de aprendizaje e investigación universitaria.
Se realizó el modelado matemático del sistema, y luego la identificación del mismo mediante datos experimentales. A partir del modelo obtenido se 
diseñó un controlador PID, el cual fue implementado mediante un microcontrolador ESP32-S3. 
Además, se realizaron análisis temporales y de respuesta en frecuencia del sistema, así como su estabilidad. Finalmente, se montó el circuito físicamente
y se logró un correcto funcionamiento del sistema de control, con un voltaje de salida estable a pesar de perturbaciones en la entrada y salida.
\vspace{-0.15cm}

\noindent \textbf{Palabras clave:} sistema de control, convertidor Buck, PID, modelado, identificación.
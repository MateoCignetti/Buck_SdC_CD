\justifying

\textbf{\textit{Resumen} - En este trabajo práctico se desarrollan varios sistemas de control sobre un convertidor buck en carácter de aprendizaje e investigación universitaria.
Se realizó el modelado matemático del sistema, y luego la identificación del mismo mediante datos experimentales. Sobre el modelo identificado,
se realizaron análisis temporales y de respuesta en frecuencia del sistema, así como su estabilidad, controlabilidad y observabilidad. Se 
diseñó un controlador PID para la teoría de control clásica y un controlador LQR con filtro de Kalman para la teoría de control moderna. 
Finalmente, se montó el circuito físicamente y se logró un correcto funcionamiento del sistema de control, con un voltaje de salida
estable a pesar de perturbaciones en la entrada y salida.}
\vspace{-0.15cm}

\noindent \textbf{\textit{Palabras clave:} sistema de control, convertidor Buck, identificación, PID, LQR, Kalman.}